\documentclass{report}
\usepackage{fontspec}
\setmainfont{[Kalpurush.ttf]}


\begin{document} 

\section{চালনা কে সি পাইলট মাধ্যমিক বিদ্যালয় (ব্যাচ ’৯৬)
আটাশ বছর পর, ফেলে আসা পঁচিশ মাসের গল্প}
\textbf{এনামুল হক হীরা,শিক্ষক, খুলনা বিশ্ববিদ্যালয়}

\noindent ভুলত্রুটি মার্জনীয় \\

১৯৯১ ক্লাস সিক্স, ১৯৯২ ক্লাস সেভেন, এবং ১৯৯৩ সালে ক্লাস এইটের মাসখানেক সব মিলিয়ে ২৫ মাস "চালনা কে সি পাইলট সেকেন্ডারি স্কুল" এবং আমার গল্প। সেই ২৫ টি সোনালী মাসের হারিয়ে যাওয়া স্মৃতি থেকে হাতড়ে কিছু রত্ন জড়ো করার চেষ্টা করছি। আজ ক্লাসের বন্ধুদের মধ্যে যাদেরকে মনে পড়ে তাদের বৃত্তান্ত নিয়ে লিখতে বসলাম। ৩০ বছর আগের কথা, তাই ভুলত্রুটি অত্যন্ত স্বাভাবিক, তবু যথাসাধ্য চেষ্টা করছি, আশা করছি সকলে লেখাটির ইমোশোনের দিকটি বিবেচনা করে ভুলত্রুটি মার্জনা করবেন।
\begin{itemize}
\item  ইকবাল আহমেদঃ (ক্লাস ফাইভ পর্যন্ত ব্লু বার্ড কিন্ডার গার্টেনে তার নাম ছিল ইকবাল মাহমুদ, এই নামটিই আমার বেশি প্রিয় ছিল) সবসময়ের বাঁধা ফার্স্ট বয়, রোল এক। ঐ বয়সি একটি বালকের আমার দেখা দ্বিতীয় সুন্দরতম হাতের লেখা (আমার বিচারে তখনও প্রথম স্থানে বটিয়াঘাটা কিশলয় স্কুলে আমার ক্লাস থ্রি তে ফেলে আসা বান্ধবী লুনা)। প্রচণ্ড বিনয়ী আর গুড বয়। কানের দুই পাশে সামান্য বড় চুল রাখতে পারলে স্টাইল করত বারবার হাত দিয়ে। ঐ বয়সে সেলুনে বাপ রা বসে থেকে চুল কাটানো তদারকি করতেন কিনা! ইকবালের বাড়ির ঠিক পাশের বাড়িতে থাকার সুযোগ হয়েছে বছর খানেক। ওদের চমৎকার বাড়িটিতে অনেক সময় কেটেছে আমার। ইকবালের বড় বোন হীরা আপু, আমার মিতা। সুন্দর সুন্দর অনেক আয়োজন ছিল ওদের বাড়িতে, ক্যারাম বোর্ড, ব্যাডমিন্টন কোর্ট, চাবি দেওয়া ট্রেন সেট (ইন্ডিয়া থেকে আনানো), রিমোট কন্ট্রোল গাড়ী (লঞ্চঘাটের মাখনের দোকান থেকে কেনা), পুকুর, বাড়ির দুই দিকেই গেটের পাশ দিয়ে মুখোমুখি বসার মত পাকা বেঞ্চ, ইত্যাদি। আমি বরাবরই পড়ালেখা বিমুখ, ফাক পেলেই বাসা থেকে পালিয়ে বের হয়ে যেতাম আম্মা যখন দুপুরে একটু ঘুমাতো তখন। ইকবালদের বাড়ির দিকে পা টিপে টিপে চলে আসতাম, তখন কপাল দোষে কয়েকবার ধরাও খেয়েছি। এসে দেখই ইকবালের আম্মা ওদের পড়াচ্ছেন, বিশেষত অংক। আমাকে দেখতে পেলে আমাকেও ডেকে বসাতেন। আমি তখন পুরাই ধরা! ঐকিক নিয়মের অংকের একটা বড় অংশ আমার এভাবে আন্টির কাছে ধরা খেয়ে শেখা।  ইকবালদের একটা জিনিস মাঝে মাঝে আমাকে ঈর্ষান্বিত করত, সেটা হলো ওদের দুই ভাইয়ের টিফিন। ওরা মাঝে মাঝে টিফিনে থানার মসজিদের মোড়ের দোকান থেকে পরটা আর ডাল কিনে খাওয়ার জন্য বরাদ্দ পেত। আর আমাকে রোজ বাসা থেকে খাবার দিত আম্মা। কেনা সেই পরটার প্রতি আমার ভীষণ আগ্রহ ছিল। ওরা অবশ্য প্রায়ই শেয়ার করত। ইকবালের সাথে টক্কর দিয়ে জীবনে একবার ফার্স্ট হয়েছিলাম ক্লাস ফাইভে। লাভ হলোনা, ক্লাস সিক্সে কেসি স্কুলে ভর্তি হয়ে নতুন রোল নাম্বার! এইচ এস সি তে আলাদা কলেজে পড়লেও কেমিস্ট্রি নবগোপাল স্যারের ব্যাচে ভোর ছয়টায় সপ্তাহে তিনদিন দেখা হত। আমার পরম সৌভাগ্য এখন আমরা একই কর্মস্থলে কর্মরত।
\item  মোহাম্মদ আলীঃ (বেশি ভদ্রতা করে মাঝে মাঝে দুইবার মোহাম্মদ বলত, মানে মোঃ মোহাম্মদ আলী)। ইকবালের বছর খানেকের বড় সহোদর ভাই। আমাদের কাছে সে ছিল একজন জ্বলজ্যান্ত হিরো। আমরা যখন বাবা-মায়ের শাসনের ভয়ের সামান্য দুষ্টামির সাহস করে উঠতে পারতাম না, সেই বয়সে আলীর দুর্দান্ত সব ডেয়ারিং কাণ্ডকারখানা আমাদের তার সাগরেদ হওয়ার উৎস ছিল। একবার তার পায়ে বড়শি ফুটে গিয়েছিল, আমরা ভয়ে ভয়ে অনেক কষ্টে চেষ্টা করেছিলাম ছাড়ানোর, যতই ছাড়াতে যাই ততই আরো বেশি করে বিঁধে যায়। শেষ পর্যন্ত ডাক্তারের কাছে যাওয়া লেগেছিল, ডাক্তার বড়শির পিছনের বল্টু কেটে ফেলে সামনের দিকে ফুটিয়ে অন্য প্রান্ত থেকে বড়শিটি বের করেছিলেন। বাজারের দিকে সম্ভবত সেই ডাক্তারের নাম ছিল ডা. হরিদাস বাবু। ছবেদ নামের একজন কর্মকার বন্ধু ছিল আমাদের। মোহাম্মদ আলীর সাথে বিশেষ খাতির। সেই সুবাদে ক্লাস ফাইভে চাইনিজ কুড়াল আর গুপ্তি বানিয়ে নিয়েছিল আলী। হারা নামের ফেড জিনস তখন নতুন এসেছে। তারই পিছনের পকেটে কুড়াল আর কোমরের দিকে বাট আর গুপ্তি! সত্যি, আমি নিজে না দেখলে এমন হিরোইক কেউ হতে পারে তা বিশ্বাস করতাম না।
৩. আশরাফুল আলম খান প্রিন্সঃ  তার বাবা (সম্ভবতঃ ডাঃ আলমগীর খান) চালনা উপজেলা স্বাস্থ্য কমপ্লেক্সের মেডিকেল অফিসার ছিলেন। সেই সুবাদে তারা হাসপাতালের কোয়ার্টারে থাকত। আমি, প্রিন্স আর আরেকজন (যার নাম এর পর আসবে) ক্লাসের দ্বিতীয়, তৃতীয়, চতুর্থ স্থান পাল্টাপাল্টি করে দখল করতাম। প্রিন্স আর তার ছোট ভাই ছোটন একসাথে হাতে সুন্দর ফ্যান্সি ছোট ছাতা হাতে হেটে স্কুলে আসত। সে সময়ে একঘেয়ে সেই বিখ্যাত শরিফ ছাতাই আমাদের প্রায় সকলের সংগ্রহে থাকত। মাঝে মধ্যেই বৌমার গাছতলা থেকে একসাথে আসার সুযোগ হয়েছে। ওদের দুই ভাইয়ের বিশেষত্ব হলো আমাদের কেসি স্কুলে কোন নির্ধারিত স্কুল ড্রেস বা ইউনিফর্ম ছিলনা, কিন্তু ওরা ওদের আগের স্কুলের (সম্ভবত ফুলতলা দামোদর স্কুল) ড্রেস পরে ক্লাসে আসত, আমার এখনও চোখে ভাসে, বাড়ন্ত শরীরে ওদের দুই ভাইয়েরই নেভি ব্লু প্যান্ট একটু খাটো হয়ে এসেছিল। প্রিন্সের ছোটভাই ছোটন নাক বন্ধ করে মুখ দিয়ে দম ছাড়ার সময় গলার কাছে (এ্যাডামস এ্যাপলে) হাতের আঙ্গুল দিয়ে আঘাত করে সাপের বীণ বাজানোর মত অদ্ভুত এক রকম শব্দ তৈরী করতে পারত। ওর মনে আছে কিনা কে জানে!
\item প্রতুল মণ্ডলঃ (নিজের নামকে সে বলার সময় বলতো পত্তুল)। ছোটখাট গড়নের নরম সময় গুড বয়। আমার কেন যেন মনে হতো ওর নিজের মধ্যে এবং বাড়িতে তার ভাল রেজাল্ট করার জন্য একটা চাপা প্রেসার আছে কিন্তু প্রকাশ করতে পারত না। আমি সত্যিই মনে মনে ভাবতাম ইকবালকে তো টপকানো যাবে না, ও আমার পজিশনটা পাক, আমি না হয় থার্ড-ফোর্থ হই। কোকড়া চুলের সহাস্য বন্ধু "পত্তুলের" সাথে ১৯৯৩ সালের পর আর দেখা হয়নি।
\item কমলেশ শীলঃ বারইখালি বাড়ি। শক্তপোক্ত গড়নের, উইং সাং ফাউন্টেন কলম ও নীল কালির ভক্ত। ভীষণ হিসেবি। কলেজ জীবণে খুলনা সিটি কলেজে শেষবার আমরা একসাথে চলেছি। হঠাৎ রেগে ফায়ার হয়ে যাওয়া কমলেশের স্বভাব ছিল, কিন্তু প্রচণ্ড কোমলমতি মনের কিশোর।
\item হান্নান সরদারঃ কেসি স্কুলের সামনের খালের প্রান্তের দিকের গ্রাম থেকে স্কুলে আসত, সম্ভবত পারিবারিকভাবে ওরা মৎসজীবী ছিল। বুধবারের হাটে মাঝে মাঝে ওকে ছোটখাট পন্য ফেরি করতে দেখেছি। তার বাম কানটি ফোটানো এবং খুব সরু একটি দুল ছিল। রহস্য জানতে চেয়ে অনুরোধ করায় বলেছিল অসুখ বিসুখের থেকে মুক্তি বা এমন কোন একটা বিশ্বাসে পারিবারিকভাবে ওটা ওকে করতে হয়েছিল। হান্নান কথা বলত খুব নিচু গলায়।
\item  কিশোর মণ্ডলঃ চালনা লঞ্চ ঘাটের কাছে ওদের বাড়ি। ওর বাবা হোমিওপ্যাথি চিকিৎসক ছিলেন। ও আমার কাছে খুব ম্যাগনেটিক ছিল, নিত্য নতুন চমক স্কুলে নিয়ে আসত। চুম্বক, লোহার গুড়ো, ছোট মটর এইসব। আর ওর মামাবাড়ি সুন্দরবনের ভিতরে হওয়ায় বিচিত্র সব জিনিস ওদের বাড়ি দেখা যেত। যেমন বানরের বাচ্চা, পোষা বেজি, ইত্যাদি। ওদের বাড়ি আমি অনেক দুপুর কাটিয়েছি, পুকুরে গোসল করেছি। কিশোরের দাদাও সম্ভবত আমাদের উপরের ক্লাসে পড়তেন। ওর কাকু বা মামা একজন গরান কাঠ দিয়ে আমাকে চমৎকার একটি গুলতির বাট বানিয়ে দিয়েছিল।
৮. গৌরঙ্গ রায়ঃ কিশোরের বাড়ির খুব কাছে একটা টেইলার্সের দোকানে কাজ করত। দোকানটা ওদের কিনা জানতাম না। আমাদের চেয়ে বেশ দীর্ঘাঙ্গী। টকটকে ফর্সা। হবেই তো, ওর নামই যে গৌরঙ্গ!ভীষণ নরম স্বাভাবের ছিল ও।
\item  মানিক মণ্ডলঃ প্রায়সই টাক থাকত। আমার কাছে আজও সে একজন সুপার হিরো, কেননা ইংরেজি ক্লাসে বাড়ির কাজ জমা না দেওয়ায় বিকাশ স্যার যত জোরে বাড়ি দিতেন তা আমাদের কারও পক্ষে সহ্য করার কল্পনা করারও সাধ্য ছিলনা, মানিক অনায়াসে বিনা বাক্যে, বিনা এক্সপ্রেশনে প্রায় রোজ এই মার খেত। শুনেছিলাম সে নৌকা বাইতো, তাই হাত অনেক শক্ত ছিল। এখন বুঝি, অনেকখানি আত্মমর্যদাও ছিল তার। ছিপছিপে মজবুত মানিকের জন্য আমার রোজ কান্না পেত! ��
১০. নবীউল ইসলামঃ ভোলার চরফ্যাসান থেকে বদলি হয়ে আসা একজন পুলিশ অফিসারের (কোর্ট জি আর ও) সন্তান। নাদুস নুদুস। কলেজের কাছে ওরা বাসা নিয়ে থাকত। আমার খুব প্রিয় বন্ধু ছিল ও। স্কুল থেকে ফেরার সময় ঝাক বেধে সবাই মিলে ফিরতাম, কিন্তু স্কুলে যাওয়ার সময় ও আমার বাসায় আসতো, কোন কোন দিন আমাদের দুইজনকে আম্মা সামান্য কিছু খেতে দিতেন, তারপর আমরা একসাথে স্কুলে যেতাম। ওর সুবিশাল সাইজের কারণে আমি বন্ধু হিসেবে নিজেকে খুব প্রোটেকটেড ভাবতাম! তবে নবীউল কুকুর ভীষণ ভয় পেত। আমাদের বাসার পাশে এজি চার্চ সংলগ্ন একটি বাড়িতে অনেকগুলি ভয়ংকর কুকুর ছিল, এটা নিয়ে নবীউল খুব চিন্তায় থাকত। একবার এলাকা পেরোতে পারলেই পগার পার ��
১১ ও ১২. গোবিন্দ হালদার ও সংকর হালদারঃ প্রায় আইডেনটিক্যাল টুইন ব্রাদার্স। পানখালি নদীর ওপার বরনপাড়ার ওদিক থেকে কমপক্ষে পাঁচ কিলোমিটার পথ হেঁটে আসতো। গলায় ছিল তুলসির মালা। সাথে কাঠের বাটের ছাতা আর বড় তিন-চারতলা টিফিন ক্যারিয়ার। টিফিনে স্ট্রেইট-কাট ভাত। স্কুলের পেছন দিকটায় পুকুরের পাড়ে বাবলা গাছের নিচে একটু উচু ঢিবির মত জায়গায় বসে আমরা টিফিন করতাম। মাঝে মধ্যে ওদের দেরি হত স্কুলে আসতে। ক্লাস টিচার জানতে চাইলে “জলদি খেয়া” মিস করেছে বা এরকম কিছু একটা বলতো।
\item ডেভিড অরবিন্দ বিশ্বাসঃ আমাদের তুলনায় সামান্য বেশি বয়সী, সম্ভবত কনভার্টেড খ্রিস্টান, খুব সম্ভবত হীড বাংলাদেশ বা সিএসএস এর কোন হোমে থাকত। স্কুলের পড়াশোনার পাশাপাশি ভোকেশনাল ক্লাস করত হোমে, সম্ভবত কার্পেন্টিং এর উপর। এর বেশি খুব একটা জানতে পারিনি বা জানা হয়ে ওঠেনি।
\item পাপিয়াঃ শিক্ষাজীবনের প্রায় পুরোটা অংশে (খুলনা জিলা স্কুল বাদে) আমার ছেলে বন্ধুর চাইতে মেয়ে বন্ধুর সংখ্যাই বেশি ছিল। কিন্তু কোনো এক বিচিত্র কারণে কে সি পাইলট স্কুলের সময় কালে খুব বেশি বান্ধবীর কথা মনে পড়ছে না। অনেক হাতড়ে পাপিয়ার কথা মনে পড়েছে যাকে “পাইপ্যে” (পেপের স্থানীয় উচ্চারণ)বলে ডাকলে ভীষণ ক্ষেপে যেত। এটুকুই!
\item ঝর্ণাঃ দ্বিতীয় যে বান্ধবীর নাম পড়ছে সে ঝর্ণা। আমাদের অনেক ছেলেদের চাইতেও অনেক লম্বা ছিল ও। বেশ সুন্দরীও বটে। আমাদের সিনিয়র ক্লাসের ভাইয়ারা আমাদের সাথে খাতির করে ওর সাথে কথা বলতে চাইতো, এটুকু মনে আছে। সাইনুসাইটিসের জন্য প্রায়ই আমার মাথা ব্যথা থাকতো, আর পকেটে টাইগার বাম থাকত। সমজাতীয় কিছু এখনও থাকে। ঝর্ণার একদিন বেশ মাথাব্যথায় টিফিন পিরিয়ডে আমার কাছ থেকে বাম নিয়েছিল, বেশ অনেকটাই কপালে লাগিয়েছিল, তারপর অনেকক্ষণ ধরে তার সে কি চোখ জ্বলা। আমার মনে আছে টিফিনের পরের ক্লাসে জ্যোতিন স্যার ইংলিশ গ্রামার ক্লাসে চলে এসেছিলেন তখনও তার জ্বালাপোড়া থামেনি।
সত্যিকারের আলোকিত এক উদ্যান
বাবার চাকরীর সুবাদে লবণাক্ত এই জনপদে আমার চারটি বছর কাটে যার মধ্যে দুই বছর আমি ব্লু বার্ড স্কুলে আর দুই বছর কেসি পাইলট স্কুলে পড়ার সৌভাগ্য হয়েছে। অনেক কিছু জেনেছি। যোগাযোগ ব্যবস্থার এতটা উন্নয়ন হয়েছে যে চালনার অনেক কর্মকর্তারা হয়তো এখন দৈনিক খুলনা থেকে যাতায়াত করেন এবং তাদের পরিবার শহরেই থাকে। আমার ক্ষুদ্র জীবনে সত্যিকারের শিক্ষা যদি কিছু পেয়ে থাকি তা শহর নয় বরং মফস্বলের জীবন থেকেই পেয়েছি। তাই বটিয়াঘাটা আর চালনা আমার প্রিয় দু’টি জনপদের নাম। স্কুলের পক্ষে আমার বন্ধু ইকবালসহ অনুজ কয়েকজন চনমনে তরুণ যে চমৎকার উদ্যোগ নিয়েছে তাকে সাধুবাদ জানাতেই হবে। আজ এতগুলো বছর পর আমি আমার ফেলে আসা জীবনের সত্যিকারের হিরোদের কথা স্মরণ করার স্বর্ণসুযোগ পেয়েছি। আমরা গুটিকয়েক অফিসার, ডাক্তার, ইঞ্জিনিয়ার, ব্যাংকারের সন্তানরা ছাড়া আমাদের প্রায় সব সহপাঠী ছিল অত্যন্ত তৃণমূল পর্যায়ের পেশাজীবীদের সন্তান। কৃষক, মজুর, জেলে, মাঝি, কামার, কুমার, কাঠমিস্ত্রী, দর্জি, ক্ষুদে দোকানী, মাছ ও কাঁকড়ার ডিপোর কর্মচারী, হকার ইত্যাদি। এবং আমার প্রায় প্রত্যেক বন্ধুকেই সেই সময় তাদের ১০/১১ বছর বয়সেই তাদের পারিবারিক পেশায় বাবা-মা কে সাহায্য করতে হয়েছে, সোজা বাংলায় উপার্জন করতে হয়েছে। সত্যি বলতে কি, আমার ভীষণ সংকোচ ও গ্লানীর সাথে উপলব্ধি হচ্ছে যে, আমরা বাসা থেকে বাটার-ব্রেড, টোস্ট, ডিম ভাজি/পোচ/সিদ্ধ, রুটি/পরটা, ভাত/খিচুড়ি ইত্যাদি গরম গরম খেয়ে বাবা মায়ের স্নেহমাখা বিদায়বাক্য শুনে স্কুলে গিয়েছি। ঠিক সেই সময় আমারই বয়েসি বন্ধুরা কেউ সকালে জাল টেনে, কেউ ক্ষেতে চারা রুয়ে, কেউ হাল চষে, কেউ দাড় টেনে, কেউ দোকানদারী করে, কেউ দর্জির কাজ সেরে হয়তো তেমন কিছু না খেয়েই স্কুলে এসেছে। ছুটির পর আমরা যখন বাসায় ফিরে কেক কিংবা পুডিং তারপর লাঞ্চ, তারপর ঘুম সেরে বিকালে খেলা করতে গিয়েছি তখন হয়তো আমার সেই বন্ধুরা দ্বিতীয় দফা আবার তাদের সেই সব উপার্জনের কাজে নেমে পড়েছে। আজ এতটা বছর পর আমার উপলব্ধি, ওরাই আমাদের সত্যিকারের হিরো, একেকজন জ্বলজ্যান্ত আলোকবর্তিকা। আমি জানিনা আমার এই লেখা আমার সেইসব বন্ধু পর্যন্ত পৌছাবে কিনা। সত্যিই ওদের দেখা পেলে আমি সবার পদধুলি নিতাম। পরিবারকে প্রথম উপার্জিত একটি টাকা তুলে দিতে আমার মত অধমের ২৬ বছর লেগেছিল, যা আমার ফেলে আসা বন্ধুরা ১০ বছরেই কাধে তুলে নিয়েছিল। আমার এই লেখার মাধ্যমে আমার সেই প্রত্যেক হিরো বন্ধুদেরকে শ্রদ্ধা জানাতে চাই। আমি জানি আপাতদৃষ্টিতে আমরা যারা একটু এগিয়ে গিয়েছি বলে সমাজ মনে করে, আমরা যদি খোজ নিই আমার বিশ্বাস আজও আমরা কেসি পাইলট স্কুলের প্রতিটি ক্লাসেই এমন অনেক গল্প পাব। আমরা মনে হয় একজোট হলে আমাদের পরের প্রজন্মের কাধের বোঝাটা একটু হালকা করে দিয়ে সহযোগিতা করতে পারি। তার মানে এই নয় যে, টাকা বা বৃত্তিই দিতে হবে, বরং এখনকার প্রজন্মকে সাহস ও গাইডলাইন দিয়েও এই কাজটি করা যেতে পারে। আমার বিশ্বাস যে দৃঢ় প্লাটফর্ম আমার প্রিয়ভাজন বন্ধু ও অনুজেরা তৈরী করেছে তা দৃঢ়তর হবে। আজীবনের জন্য অটুট থাকবে।
এত দীর্ঘ লেখাটি ধৈর্য ধরে পড়ার জন্য আন্তরিক ধন্যবাদ ও অকৃত্রিম কৃতজ্ঞতা।
\end{itemize}
\end{document}